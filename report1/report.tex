\documentclass[a4j]{jarticle}
\title{知能情報メディア実験中間レポート \\ \ 複数話者の同時発話音声からの個別音声の抽
出}
\author{情報科学類, 3年, 2クラス, 学籍番号:201311403 \ 山岡 洸瑛, Yamaoka
Kouei}
\date{2015/5/15(金)}

%余白設定
\setlength{\topmargin}{20mm}
\addtolength{\topmargin}{-1in}
\setlength{\oddsidemargin}{20mm}
\addtolength{\oddsidemargin}{-1in}
\setlength{\evensidemargin}{15mm}
\addtolength{\evensidemargin}{-1in}
\setlength{\textwidth}{170mm}
\setlength{\textheight}{254mm}
\setlength{\headsep}{0mm}
\setlength{\headheight}{0mm}
\setlength{\topskip}{0mm}


\def\pgfsysdriver{pgfsys-dvipdfmx.def}

\usepackage{ascmac}
\usepackage{here}
\usepackage{tikz}
\usetikzlibrary{trees}
\thispagestyle{empty}

\usepackage{listings,jlisting}
\renewcommand{\lstlistingname}{リスト}
\lstset{
basicstyle=\ttfamily\scriptsize,
commentstyle=\textit,
classoffset=1,
keywordstyle=\bfseries,
frame=tRBl,
framesep=5pt,
showstringspaces=false,
numbers=left,
stepnumber=1,
numberstyle=\tiny,
tabsize=2
}

\begin{document}
 \maketitle
\section*{現在までの進捗状況}
\subsection*{課題1: 瞬時混合ICAの実装}
課題1はすべて実装した.ここで学んだのは,混合の単純化されたモデルである
瞬時混合であり,これに関しては大まかには理解できたと考えている.事実ここ
で得た知識が課題2において非常に役立った.ここで得た知識とは,音声の混合
及び分離においてどのようなモデルで表すことができ,どのような式で表現でき
るのか,ということと,その理由である.しかし,分離行列の推定方法など理解
しきれていない部分もあり,そういった部分の理解が今後の課題である.1,2回
授業資料を読んだ程度では理解出来なかったため,課題が進み時間に余裕ができ
たら手計算で確認してみようと考えている.

\subsection*{課題2-1: たたみ込み混合の分析}
課題2はまだ1つ目の課題すら終わっていない.しかし,追加で出された畳み込み
信号をつくる課題は完了した.ここでは畳み込み積分やインパルス応答など
まだ知らなかった多くの知識を必要とし,実装が難しかったが,TAの杉本さんな
どに教えてもらいながら最終的には実装することができた.実装した今では知ら
なかった知識も得ることができ,理解することもできたと考えている.このレポー
トを書いている頃には,他の授業でもフーリエ変換や畳み込み積分を本格的に扱
いだし,より理解も深まっている.

\section*{現在の問題点}
現在取り組んでいる課題は課題2-1-1である.この課題は畳み込み混合信号の
stft分析と逆stft分析を行う課題である.stftについては理解し,そのために必
要な手順も理解しているが,それを実装することに手間取っている.手間取って
いる点は,stft分析し,推定された音源信号を時間領域に戻す点である.stft分
析の過程でハミング窓を使用し,フレームシフトをフレーム長の半分にした場合
など,条件を固定してやれば実装することはできると考えているが,フレームシフトをフレー
ム長の$\frac{1}{4}$に変更した場合などに動かない,汎用性の低いコードになって
しまう.窓については窓関数だけを変えれば動くような,フレームシフトについ
てはフレーム長の半分,$\frac{1}{4}$どちらにも,あるいは可能ならばその他の一般的
でない長さでも対応できるようなコードを作成したいと考えている.\\\ \ 
具体的には音声信号と同じ長さの配列を用意し,窓関数で切り出すたびに同じ
インデックスに1加算する.これによって音声信号のどの位置を何回切り出した
かを保存することができる.最後の段階で切り出された音声信号に逆fftをし,足しあわ
せると,オーバーラップにより2回以上切り出された部分の振幅が,切り出された
回数倍されてしまうが,この保存用配列で割ることで元の信号と同じ振幅に復元
することができる,と考えている.\\ \ \ 
このように実装方針は立っているので後は実装するだけであるが,うまく行って
いないのが現状である.従ってまずは汎用性の低いコードを作成し,それを拡張
していく,という方針で今後は動いていく予定である.






\end{document}
